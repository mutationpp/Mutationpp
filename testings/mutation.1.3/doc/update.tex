--------------
|mutation.1.1|
--------------
12-10-2004 (Olivier Chazot)
sources/general/mutation.f
XTOL was used instead of X when calling the routines:
  CALL COLLISION (WR1, LWR1, WR2, LWR2, TH, TE, ND, X)
  CALL LAMBDAE (WR1, LWR1, WR2, LWR2, X, TE, LAMBDATE, 3)
  CALL SIGMAE (WR1, LWR1, WR2, LWR2, X, TE, SIGMA, 2)
yielding to wrong values of the electrical conductivity.
----------
12-16-2004 (Thierry Magin)
sources/general/initialize.f
sources/thermo/energy.f
sources/thermo/enthalpy.f
sources/thermo/entropy.f 
sources/thermo/gibbs.f
Allow for species without any electronic levels (if they are 
neglected or in the case of the proton Hp). 
----------
12-16-2004 (Thierry Magin) 
sources/general/length.f
sources/general/initialize.f
sources/thermo/energy.f
sources/thermo/enthalpy.f
sources/thermo/entropy.f 
sources/thermo/gibbs.f
Rotational d.o.f. of non linear polyatomic molecules.
----------
11-09-2006 (Janos Molnar and Marco Panesi)
sources/chemistry/arrhenius.f
The expression of the equilibrium constant has been corrected 
(missing LOG(R) term).
----------
05-29-2007 (Thierry Magin)
sources/chemistry/arrhenius.f
The subroutine DTRANSLATE did not allow for the 3rd decimal to be
read adequately.
--------------
|mutation.1.2|
--------------
11-10-2006 (Thierry Magin)
sources/general/length.f
sources/general/mutation.f
sources/general/shock.f
sources/thermo/enthalpy.f
The number of vibrational temperatures is given by the routine length.
The vibrational temperature become  a vector in the thermodynamic 
routines. The mutation and shock programms are modified accordingly.
----------
10-01-2007 (Olaf Marxen)
data/thermo/NO
The degeneracy of NO(X^2\Pi_r) has been corrected (4 instead of
2).
----------
